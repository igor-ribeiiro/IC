\documentclass[conference]{IEEEtran}
\IEEEoverridecommandlockouts
% The preceding line is only needed to identify funding in the first footnote. If that is unneeded, please comment it out.
\usepackage{cite}
\usepackage{amsmath,amssymb,amsfonts}
\usepackage{algorithmic}
\usepackage{graphicx}
\usepackage{textcomp}
\usepackage{dsfont}
\usepackage{xcolor}
\usepackage[utf8]{inputenc}
\usepackage{url}
\usepackage[export]{adjustbox}
\def\BibTeX{{\rm B\kern-.05em{\sc i\kern-.025em b}\kern-.08em
    T\kern-.1667em\lower.7ex\hbox{E}\kern-.125emX}}
\begin{document}

\title{Text Classification for Twitter Sentiment\\
}

\author{\IEEEauthorblockN{1\textsuperscript{st} Igor Mourão Ribeiro}
\IEEEauthorblockA{\textit{Computer Engineering Departament} \\
\textit{Instituto Tecnológico de Aeronáutica}\\
São José dos Campos, Brazil \\
igormr98mr@gmail.com}
\and
\IEEEauthorblockN{2\textsuperscript{nd} Isabelle Ferreira de Oliveira}
\IEEEauthorblockA{\textit{Computer Engineering Departament} \\
\textit{Instituto Tecnológico de Aeronáutica}\\
São José dos Campos, Brazil \\
isabelle.ferreira3000@gmail.com}
\and
\IEEEauthorblockN{3\textsuperscript{rd} José Luciano de Morais Neto}
\IEEEauthorblockA{\textit{Computer Engineering Departament} \\
\textit{Instituto Tecnológico de Aeronáutica}\\
São José dos Campos, Brazil \\
zluciano.t19@gmail.com}
%\and
%\IEEEauthorblockN{4\textsuperscript{th} Thiago Filipe de Medeiros}
%\IEEEauthorblockA{\textit{Computer Engineering Departament} \\
%\textit{Instituto Tecnológico de Aeronáutica}\\
%São José dos Campos, Brazil \\
%thiagoofic@gmail.com}
}

\maketitle


\begin{abstract}
Esse relatório documenta a implementação de algoritmos de Processamento de Linguagem Natural, aplicando diferentes técnicas de Machine Learning para classificar Tweets entre sentimentos positivos e negativos. Os algoritmos implementados foram Naive Bayes e Support Vector Machine, utilizando diferentes features produzidas a partir de um dataset do Kaggle, e comparando-os pelas métricas de acurácia e coeficiente Kappa.
\end{abstract}

\begin{IEEEkeywords}
Processamento de Linguagem Natural, Naive Bayes, Aprendizagem Supervisionada, Support Vector Machine
\end{IEEEkeywords}

\section{INTRODUCTION}
Um dos aspectos relevantes da interação entre as pessoas na atualidade é a expressão de sentimentos por meio de textos nas mídias sociais. Nesse contexto, o monitoramento das redes sociais pode ser explorado como forma de extrair a aceitação e/ou aprovação de produtos e também obter conhecimento dos usuários. A análise de sentimentos surge da necessidade de tratar e interpretar textos, opiniões e comentários realizados pelos usuários em redes sociais. Por meio das informações subjetivas extraídas textos em linguagem natural, pode ser gerado conhecimento estruturado, auxiliando a tomada de decisão. 

A expansão da Internet e a utilização das redes sociais definiram um ecossistema de interação, no qual os usuários deixaram de ser receptores passivos e se tornaram produtores, compartilhadores e avaliadores de conteúdo. Em um cenário onde as reputações de empresas e a aceitabilidade de produtos no mercado são diretamente afetadas pela repercussão de opiniões de seus clientes na web, tanto quanto pelas campanhas de publicidade, a análise de sentimentos surge como um diferencial para rastreamento do conteúdo emocional daquilo que se escreve e compartilha nas redes sociais. Nesse sentido, a análise de sentimentos alia-se à publicidade promovendo subsídios para definição de estratégias e garantia da vantagem competitiva.

A análise de sentimentos ser aplicada na gestão de informação por exemplo, fornecendo feedback do cliente a partir do conteúdo dos diversos canais de comunicação e entregando informações úteis para tomada de decisão e definição de estratégias para satisfação dos clientes.

O Twitter\cite{twitter} é uma rede social e servidor para microblogging muito utilizada, que será utilizada para essa análise de sentimentos. Atualmente, o limite máximo de um \textit{tweet} (mensagem postada no blog) é de 280 caracteres e tem-se um total de 6000 \textit{tweets} por segundo o que implica em 200 bilhões por ano. Exemplos de \textit{tweets} com emoções podem ser vistos nas Figuras \ref{fig:dataset_positivo} e \ref{fig:dataset_negativo}.

\begin{figure}[htbp]
	\includegraphics[width=0.5\textwidth,center]{imgs/tweet_positivo.png}
	\caption{\textit{Tweet} com mensagem positiva}
	\label{fig:dataset_positivo}
\end{figure}


\begin{figure}[htbp]
	\includegraphics[width=0.5\textwidth,center]{imgs/tweet_negativo.png}
	\caption{\textit{Tweet} com mensagem negativa}
	\label{fig:dataset_negativo}
\end{figure}

Neste sentido, este trabalho tem como objetivo apresentar a análise de
sentimentos aplicada a textos em linguagem natural de uma rede social, usando diversos modelos e entradas para posterior comparação.

\section{Results and Discussion}

\section{Conclusion}

Pode-se concluir através dos resultados que o SVM, mesmo exigindo um processamento muito grande e trabalhando com 10\% do tamanho padronizado de dados, foi o que teve o pior resultado, sendo assim um classificador ruim se comparado aos demais.

Sendo assim, o método de \textit{Naive-Bayes} foi consideravelmente se baseando nos valores de Kappa obtidos. Sendo o método em \textit{Word-Level} ligeiramente melhor que os demais, embora todos, tenham tido um desempenho similar, com exceção do método em \textit{Char-Level} que teve um desempenho um pouco menor entre os que foram feitos baseados em \textit{Bayes}.

\begin{thebibliography}{00}
\bibitem{dataset_completo} Dataset de Sentimentos, \url{https://www.kaggle.com/kazanova/sentiment140}.
\bibitem{twitter} Twitter, \url{https://twitter.com}
\bibitem{kibriya} Kibriya, Ashraf \& Frank, E. \& Pfahringer, Bernhard \& Holmes, Geoffrey. (2004). Multinomial naive Bayes for text categorization revisited. Advances in Artificial Intelligence. 488-499. \label{kibriya}
\bibitem{https://scikit-learn.org/stable} SciKit-Learn, \url{https://scikit-learn.org/stable/}
\bibitem{kappa} Cohen, Jacob (1960). "A coefficient of agreement for nominal scales". Educational and Psychological Measurement. 20 (1): 37–46. \url{doi:10.1177/001316446002000104}
\bibitem{svm} Implementing SVM and Kernel SVM with Python's Scikit-Learn \url{https://stackabuse.com/implementing-svm-and-kernel-svm-with-pythons-scikit-learn/}
\bibitem{cohen_score} sklearn.metrics.cohen\_kappa\_score \url{https://scikit-learn.org/stable/modules/generated/sklearn.metrics.cohen_kappa_score.html}

\end{thebibliography}

\end{document}
